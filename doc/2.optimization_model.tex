% Created 2020-05-02 Sat 00:00
% Intended LaTeX compiler: pdflatex
\documentclass[11pt]{article}
\usepackage[utf8]{inputenc}
\usepackage[T1]{fontenc}
\usepackage{graphicx}
\usepackage{grffile}
\usepackage{longtable}
\usepackage{wrapfig}
\usepackage{rotating}
\usepackage[normalem]{ulem}
\usepackage{amsmath}
\usepackage{textcomp}
\usepackage{amssymb}
\usepackage{capt-of}
\usepackage{hyperref}
\author{Dr. Shuai Wang}
\date{09/28/2019}
\title{Cincinnati Public School Bus Routing and Bell Time Optimization}
\hypersetup{
 pdfauthor={Dr. Shuai Wang},
 pdftitle={Cincinnati Public School Bus Routing and Bell Time Optimization},
 pdfkeywords={},
 pdfsubject={},
 pdfcreator={Emacs 26.3 (Org mode 9.3.6)}, 
 pdflang={English}}
\begin{document}

\maketitle
\tableofcontents


\section{Overview}
\label{sec:org2f91e07}
\subsection{Single school}
\label{sec:orgd7ce842}

We first assign the stops, which optimize the number of stops for each school and the total student walking distance.
This is similar to a facility location problem. 

Single school routing:

\begin{enumerate}
\item Generate a set of single feasible trip using randomized greedy heuristic, making sure each stop is served by a nonempty set of feasible trips.(func: greedy)
\item Run the heuristic N times, so that each stop is covered by N different feasible trips. (func: generateRoutes)
\item Find the best sub set of trips that covers each stop, by using a minimum cover problem. Some stops may be servered by more than one trip. (func: bestRoutes)
\item Trim the stops covered with multiple trips. (func: buildSolution)
\end{enumerate}

The output of single school routing is set of trips that covers every stop
excatly once for each school.

The senarios are the tradeoff parameters \(\lambda\) to obtain an array of varied
routing scenarios for each school.

For each school, we end up with a set of routing scenarios, where each scenario
is a complete set of trips that covers every stop exactly once.

Each senario is located in a different region of the Pareto front between two objectives,
the number of buses and the average student travel time.

\subsection{Multiple schools}
\label{sec:orga4d74b9}

\subsubsection{Scenario selection}
\label{sec:orgc733fec}

The input is a set of scenarios Rs of a size hs for each school S. Because what
is optimal for one school may not be optimal for the entire system. We consider
several several scenarios on the Pareto frontier of two objectives: numer of
buses and average riding time. The trade-offs makes sense because shorter routes
are more easily connected into bus schedules.

Then we first jointly select one scenario for each school in a way that favors
maximal re-use of buses from school to school, by using a network flow stucture
that minimized the objective across the whole district: \textbf{\textbf{minimum number of buses
out of Yard}}.

Notes:
This only tells which scenario between the school can be linked, namely some
"Departure nodes" from school1 can be linked to some "Arrival nodes" from the
school2. It \textbf{\textbf{does not}} tell which specific trips are linked to the school2.
On the contrary, it might have multiple trips from same scenario of school1 can
be linked to school2.

\subsubsection{Bus Scheduling}
\label{sec:orgd17de5d}

Given the one routing scenario for each school, we can then solve another
integer optimization problem to identify a trip-by-trip iternary for each bus on
the fleet. In this final subproblem, we optimize the numeber of buses jontly in
the morning and in the afternoon.

Notes:
The input of the Bus selection is an array of scenario for each school.
The trips in the scenario can have multiple cases where trip can be linked from
the school1 and school2. Consequently, bus scheduling help to choose the minimum
trips which linking the school1 and school2. E.g. scenario1 from school1 has 3
different trips \{2,12,39\} can be linked to school2's scenario2 trip \{32\}:
\{2,12,39\} -> \{32\}, the bus scheduling will only keep one and only one from
\{2,12,39\} to be linked to \{32\}.


\section{Routing: Bi-objective Routing Decomposition (BiRD)}
\label{sec:orgf83542a}
Decomposing the routing problem:

\begin{enumerate}
\item single-school

\begin{enumerate}
\item stop assignment
\item single-school routing
\end{enumerate}

\item multi-school

\begin{enumerate}
\item scenario selection
\item bus scheduling
\end{enumerate}
\end{enumerate}

\subsection{Stop Assignment}
\label{sec:org9cb1c70}
\subsubsection{Set/Parameters}
\label{sec:org9a8f920}
\begin{center}
\begin{tabular}{lll}
Identifier & Index/Set Domain & Comment\\
\hline
schools & S & set of schools\\
students & P & set of students\\
\(P_s\) & \(s \in S\) & set of students attending school s\\
Locations & C & set of all locations can serve as bus stops\\
\(C_p\) & \(C_p \subseteq C\) & Allowed bus stops for student \(p \in P_s\)\\
\(d_{p,c}\) &  & walking distance for student p to a stop \(c \in C_p\)\\
\end{tabular}
\end{center}

For student p with special needs may require a "door-to-door" pickup:
the set \(C_p\) is a singleton \({c}\), where \(d_{p,c} = 0\).

The objective is to minimize the number of stops for each schools and
the total walking distance. This is similar to a facility location
problem.

\subsubsection{Variable}
\label{sec:org6ed95a1}
\(z_c\): binary whether stop c is selected for school S.

\(y_{p,c}\): binary whether student p is assigned to stop c. 

\subsubsection{Objective and Constraints}
\label{sec:org02ad4fa}
For each school S, minimize the total number of stops and walking distance:

$$ min \sum_{c \in C} z_c + \beta \sum_{p \in P_s} \sum_{c \in C_p} d_{p,c} y_{p,c}$$

subject to:

\begin{enumerate}
\item student p is assigned to c only if stop c is selected for school s.
$$y_{p,c} \leq z_c,\quad  \forall p \in P_s , c \in C_p$$

\item Each student is assigned to 1 stop
$$\sum_{c \in C_p} y_{p,c} = 1 ,\quad \forall p \in P_s $$
\end{enumerate}

Where \(\beta\) is the trade-off of the total number of stops and walking
distance.

\subsection{Single school routing}
\label{sec:orgd7e2d6f}
After the stop assignment, \(C_s\) is the set of bus stops which serve
school S.
\(n_{c,S}\) is number of students from school S at a stop c.

\subsubsection{Bus}
\label{sec:orgb141c6a}
\begin{center}
\begin{tabular}{lll}
Identifier & index/set & comment\\
\hline
Bus Type & b/B & \\
Capacity \(Q_b\) & b & \\
Wheelchair spots \(W_b\) & b & fixed number of wheelchair\\
Bus depot \(Y\) &  & \textbf{stored during the night or day???}\\
\end{tabular}
\end{center}

\subsubsection{Trip}
\label{sec:orgef49fb8}
\begin{enumerate}
\item Time
\label{sec:org90c0f1a}
\begin{center}
\begin{tabular}{ll}
Identifier & comment\\
\hline
\(t_{c,S}^{pick}\) & length of time to pick all p for S at c; function of \(n_{c,S}\)\\
\(t_{S}^{drop}\) & length of time to drop all p at school S; fixed\\
\(t_{l_1,l_2,\tau}^{drive}\) & driving time from \(l_1\) to  \(l_2\) \(\in S \cup C \cup Y\) at time \(\tau\)\\
\(\tau_S\) & all buses must arrive at school S at \(\tau_S\)\\
\(t_{S}^{drop}\) + \(\tau_S\) & bus free to leave at this time\\
\(T^{max}\) & maximum time that students allowed to spend on the bus\\
\end{tabular}
\end{center}

\(t_{l_1,l_2,\tau}^{drive}\) are deterministic and known, and usual got from API
such as Google Map. Due to reality, such as accident, and to mitigate the impact
of the traffic, the drop-off times \(t_{S}^{drop}\) is artificially increase to
say 10-15 minutes even it only takes 3 minutes.

\item Trip/Route
\label{sec:orga494c8f}
A trip/route is a ordered sequence of stops visited or served, by a bus. 
For school S, a feasible trip/route:
\begin{enumerate}
\item no student spends more than \(T^{Max}\) between pickup and drop-off.
\item there exists a type of bus with capacity and carry all students assigned to
the stops severed by the trip.
\end{enumerate}

Given a feasible route \(R\):

\begin{itemize}
\item \(B_R \subseteq B\): the set of types of buses that have the necessary capacity to serve the trip.

\item \(T_R\): service time of the trip, the time between arrival at the stop to the final school

\item \(T_S\): A set of feasible trips, each stops \(c \in C_s\) is served by a nonempty 
set of feasible trips \(T_c \subset T_S\).

For example:
School S has 4 different sets of feasible trips, stop c4 can be served by two
differents trips 3 and 5, so \(T_c = {3,5}\).

The heuristic returns a set of trips T what covers each stop \textbf{\textbf{exactly once}}.
Then we run it N times to build a set of feasible trips where each stop is
covered by serveral trips. The heuristic returns N different solutions and
each stop will be covered by N different feasible trips.
\end{itemize}

Trips in \(T_S\) indexed by \(1,...,m_S\).

For given set of trips \(T \subseteq T_S\): \(I(T)\) is the subset of
\(\{1,...,m_S\}\) corresponding to trips in T.

For example:
School S has 4 different set of feasible trips, with 25 different trips in total.
Stop c4 can be served by two differents trip 3 and 5, so \(I(T_c) = I(T_4) = {3,5}
\subset {1,...25}\)
\end{enumerate}

\subsubsection{Find best set of routes}
\label{sec:org7c4c3fe}
\begin{enumerate}
\item Variable
\label{sec:org54dc371}
\(r_i\): binary whether trip i is selected \(\forall i \in \{1,...,m_S\}\)

\item Objective
\label{sec:org693adc9}
$$ min \hspace{0.5cm} \lambda \sum_{i=1}^{m_S} r_i + \sum_{i=1}^{m_S} r_i \Theta_i $$

\item Constraints
\label{sec:org3db8b53}
 every stop must be served by at least one trip. 
$$\sum_{i \in I(T_c)} r_i \ge 1, \quad \forall c \in C_S$$ 

where, for each trip \(i\) in \(T_s\), \(\Theta_i = \sum_{p:c_p \in C_i}
     \theta_p^{(i)}\), where \(C_i\) reprents the list of stops served by the
\(i-th\) trip \(i\),

\(\theta_p^{(i)}\) corresponds to tiem spent by student \(p\) on the bus druing
trip \(i\).

\(\lambda\) controls the importance of the number of trips relative to the
total time students on the bus.

In the optimal solution above, some stops may be served by more than one trip. 
Set of trips \(T_c^{opt} \subset T_S\) serving each such stop c. For each
\(i\in I(T_c^{opt})\), we compute the improvement \(\delta_i\) to the objective
that would be obtained by removing stop \(c\) from trip \(i\).
\end{enumerate}






\subsection{Scenario Selection}
\label{sec:orga54f9ea}
scenario selection graph with \(\{{\mathcal{N},\mathcal{A}\}}\) 
where \(\mathcal{N}\) consists of nodes: 
\subsubsection{Node}
\label{sec:org5fbcf1b}

\begin{center}
\begin{tabular}{lll}
Type & Notation & comment\\
\hline
depot & y & \\
trip & \(\rho_{S,h,R}\) & for each school S, and routing scenario \(1 \le h \le h_S\) and each trip \(R \in \mathcal{T}_S^h\)\\
available node & \(a_{S,h}\) & for each school S, and routing scenario \(1 \le h \le h_S\)\\
\end{tabular}
\end{center}

\subsubsection{Arcs}
\label{sec:orgf991e20}
The set of arcs \textbf{\textbf{\(\mathcal{A}\)}}  includes an arc from depot to trip; trip to corresponding
available node; from available node back to the depot.

Also, an arc from each available node to each trip node where trip \(R' \in
  \mathcal{T}_{S'}^{h'}\) is compatible with the a bus staring from school S.
\(\tau_S + t_S^{drop-off} + t_{S, c_{R'}^{start}, \tau_S}^{drive} + \mathcal{T}_R \le \tau_{S'}\)

For a note \(i \in \mathcal{N}, let \hspace{0.1cm} \mathcal{I}{(i)} \subseteq \mathcal{N}\) be the
in-neighborhood of the node i, and \(\mathcal{O}{(i)} \subseteq \mathcal{N}\)
\subsubsection{Variable}
\label{sec:orgaf8363d}
\begin{enumerate}
\item \(f_i^j\): integer flow variables for each arc(i,j).
\item \(\mathcal{Z}_{S,h}\): whether to select scenario h for school S or not
\end{enumerate}

\subsubsection{Objective}
\label{sec:org34a4d45}
Minimize the total number of buses corresponds to minimizing the total flow
out of the yard node y.
$$ minimize \sum_{s \in S}\sum_{h=1}^{h_S}\sum_{R \in \mathcal{T}_S^h}f_y^{\rho_{S,h,R}}$$ 

\subsubsection{Constraints}
\label{sec:orgff48691}
flows along all arcs must be integral.
\begin{enumerate}
\item $$f_{a_{S,h}}^{\rho_{S,h,R}}= z_{S,h},  S \in \mathcal{S}, 1\leq h \in h_s, R \in \mathcal{T}_h^S$$
\item $$\mathcal{Z}_{S,h} \in \{0,1\}, \quad \forall S \in S, 1 \le h \le h_S$$
\item $$\sum_{h=1}^{h_S}Z_{S,h} = 1, \quad \forall S \in \mathcal{S}$$
\item $$\sum_{i \in \mathcal{I}{(i)}}f_j^i  = \sum_{j \in \mathcal{O}{(i)}}f_i^j,
         \quad \forall i \in \mathcal{N}$$
\item $$f_{i,j} \in \mathbb{Z}_+ \quad  \forall (i,j) \in \mathcal{A} $$
\end{enumerate}

The binary variable \(Z_{S,h}\) is 1 if \(\mathcal{T}_S^h\) is the selected set
of trips for school S.


\(3\) exactly one set of trips/one scenario is
selected for each school. 

\(4\) ensures conservation of flow \(f_i^j\) at each node, 
\begin{itemize}
\item At the depot node \(y\), it means that buses leaving the depot must
eventually come back.
\item At a trip node \(\rho_{S,h,R}\), it means that a bus serving trip \(R\) must then
become available at school \(S\) at time \(\tau_S + t_S^{drop-off}\)
\item At an availability node \(a_{S,h}\) , it means that a bus that is available at
a school after serving a trip must either return to the yard or serve
another trip.
\end{itemize}

\subsection{Bus Scheduling}
\label{sec:orgf39c523}
To determine the final bus schedules given exactly one routing routing scenario for each school.
We define a "bus selection graph" (\(\overline{\mathcal{N}}, \overline{\mathcal{A}}\)), where the
set of nodes \(\overline{\mathcal{N}}\) consists of: 
\begin{enumerate}
\item a node \(y_{b,\mathcal{l}}\) for each bus type \(b \in \mathcal{B}\) and each
bus depot locations \(\mathcal{l} \in \mathcal{L}\).
\item a trip node \(\rho_{S,R,b,l}^{AM}\) for each school S, each morning trip R in the selected
scenario, each bus type b, and each depot location l
\end{enumerate}

and set of arc  \(\overline{\mathcal{A}}\) consists of:
\begin{enumerate}
\item an arc to and from the depot node \(y_{b,l}\) for each trip node
\(\rho_{S,R,b,l}^{AM}\) and \(\rho_{S,R,b,l}^{PM}\)
\item an arc from each trip node \(\rho_{S,R,b,l}^{AM}\) and
\(\rho_{S,R,b,l}^{PM}\) where trip \(R^{'}\) for school \(S^{'}\) is
time-compatible with a bus starting from school \(S\).
\end{enumerate}

Each arc(i,j) has a cost(i,j), and bus cost(b) for each bus type.

\subsubsection{Variables}
\label{sec:org38136f9}
\begin{enumerate}
\item BusFlow \(f_i^j\)  all the edges for the graph. Binary
\item YardCapacity \(K_{b,l}\) positive integer
\end{enumerate}
\subsubsection{Objective}
\label{sec:org07821ea}
$$ minimize   \quad \gamma \sum_{b \in \mathcal{B}} \sum_{l \in \mathcal{L}}c_b K_{b,l} + (1 - \gamma) \sum_{(i,j) \in \mathcal{\overline{A}}}c_{i,j}f_i^{j}$$ 
\subsubsection{Constraints}
\label{sec:orgff9d5a9}
\begin{enumerate}
\item $$\sum_{i \in \mathcal{\overline{I}}{(i)}}f_j^i  = \sum_{j \in \mathcal{\overline{O}}{(i)}}f_i^j,
      \quad  \forall i \in \mathcal{\overline{N}}$$
enforces flow conservation, which simply means that buses must leave the
 yard, serve at least one trip, and then return to the yard.

\item $$\sum_{b \in \mathcal{B}} \sum_{l \in \mathcal{L}} \sum_{i \in \mathcal{\overline{I} (\rho_{R,S,b,l}^{q})}} f_i^{\rho_{R,S,b,l}^{q}} = 1, \quad \forall s \in S, \forall R \in \mathcal{T}_S^{q}, q \in {\{AM, PM\}} $$

enforces that each trip is served by exactly one bus, from one particular
depot location and of one particular type.

\item $$\sum_{i \in \mathcal{O^{q(y_{b,l})}}} f_y{b,l}^i \leq K_{b,l}, \quad \forall b \in
     \mathcal{B}, l \in \mathcal{L}, q \in {\{AM, PM\}}$$
\end{enumerate}
\end{document}
