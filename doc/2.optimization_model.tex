
\section{Routing: Bi-objective Routing Decomposition (BiRD)}
\label{routing-bi-objective-routing-decomposition-bird}
Decomposing the routing problem:

\begin{enumerate}
\item single-school

\begin{enumerate}
\item stop assignment
\item single-school routing
\end{enumerate}

\item multi-school

\begin{enumerate}
\item scenario selection
\item bus scheduling
\end{enumerate}
\end{enumerate}

\subsection{Stop Assignment}
\label{stop-assignment}
\subsubsection{Set/Parameters}
\label{setparameters}
\begin{center}
\begin{tabular}{lll}
Identifier & Index/Set Domain & Comment\\
\hline
schools & S & set of schools\\
students & P & set of students\\
\(P_s\) & \(s \in S\) & set of students attending school s\\
Locations & C & set of all locations can serve as bus stops\\
\(C_p\) & \(C_p \subseteq C\) & Allowed bus stops for student \(p \in P_s\)\\
\(d_{p,c}\) &  & walking distance for student p to a stop \(c \in C_p\)\\
\end{tabular}
\end{center}

For student p with special needs may require a "door-to-door" pickup:
the set \(C_p\) is a singleton \({c}\), where \(d_{p,c} = 0\).

The objective is to minimize the number of stops for each schools and
the total walking distance. This is similar to a facility location
problem.

\subsubsection{Variable}
\label{variable}
\(z_c\): binary whether stop c is selected for school S.

\(y_{p,c}\): binary whether student p is assigned to stop c. 

\subsubsection{Objective and Constraints}
\label{objective-and-constraints}
For each school S, minimize the total number of stops and walking distance:

$$ min \sum_{c \in C} z_c + \beta \sum_{p \in P_s} \sum_{c \in C_p} d_{p,c} y_{p,c}$$

subject to:

\begin{enumerate}
\item student p is assigned to c only if stop c is selected for school s.
$$y_{p,c} \leq z_c,  \forall p \in P_s , c \in C_p$$

\item Each student is assigned to 1 stop
$$\sum_{c \in C_p} y_{p,c} = 1 ,\forall p \in P_s $$
\end{enumerate}

Where \(\beta\) is the trade-off of the total number of stops and walking
distance.

\subsection{Single school routing}
\label{single-school-routing}
After the stop assignment, \(C_s\) is the set of bus stops which serve
school S.
\(n_{c,S}\) is number of students from school S at a stop c.

\subsubsection{Bus}
\label{bus}
\begin{center}
\begin{tabular}{lll}
Identifier & index/set & comment\\
\hline
Bus Type & b/B & \\
Capacity \(Q_b\) & b & \\
Wheelchair spots \(W_b\) & b & fixed number of wheelchair\\
Bus depot \(Y\) &  & \textbf{stored during the night or day???}\\
\end{tabular}
\end{center}

\subsubsection{Trip}
\label{sec:org82671d1}
\begin{enumerate}
\item Time
\label{sec:orgb6ae6bc}
\begin{center}
\begin{tabular}{ll}
Identifier & comment\\
\hline
\(t_{c,S}^{pick}\) & length of time to pick all p for S at c; function of \(n_{c,S}\)\\
\(t_{S}^{drop}\) & length of time to drop all p at school S; fixed\\
\(t_{l_1,l_2,\tau}^{drive}\) & driving time from \(l_1\) to  \(l_2\) \(\in S \cup C \cup Y\) at time \(\tau\)\\
\(\tau_S\) & all buses must arrive at school S at \(\tau_S\)\\
\(t_{S}^{drop}\) + \(\tau_S\) & bus free to leave at this time\\
\(T^{max}\) & maximum time that students allowed to spend on the bus\\
\end{tabular}
\end{center}

\(t_{l_1,l_2,\tau}^{drive}\) are deterministic and known, and usual got from API
such as Google Map. Due to reality, such as accident, and to mitigate the impact
of the traffic, the drop-off times \(t_{S}^{drop}\) is artificially increase to
say 10-15 minutes even it only takes 3 minutes.

\item Trip/Route
\label{sec:orge254669}
A trip/route is a ordered sequence of stops visited or served, by a bus. 
For school S, a feasible trip/route:
\begin{enumerate}
\item no student spends more than \(T^{Max}\) between pickup and drop-off.
\item there exists a type of bus with capacity and carry all students assigned to
the stops severed by the trip.
\end{enumerate}

Given a feasible route \(R\):

\begin{itemize}
\item \(B_R \subseteq B\): the set of types of buses that have the necessary capacity to serve the trip.

\item \(T_R\): service time of the trip, the time between arrival at the stop to the final school

\item \(T_S\): A set of feasible trips, each stops \(c \in C_s\) is served by a nonempty 
set of feasible trips \(T_c \subset T_S\).

For example:
School S has 4 different sets of feasible trips, stop c4 can be served by two
differents trips 3 and 5, so \(T_c = {3,5}\).

The heuristic returns a set of trips T what covers each stop \textbf{\textbf{exactly once}}.
Then we run it N times to build a set of feasible trips where each stop is
covered by serveral trips. The heuristic returns N different solutions and
each stop will be covered by N different feasible trips.
\end{itemize}

Trips in \(T_S\) indexed by \(1,...,m_S\).

For given set of trips \(T \subseteq T_S\): \(I(T)\) is the subset of
\(\{1,...,m_S\}\) corresponding to trips in T.

For example:
School S has 4 different set of feasible trips, with 25 different trips in total.
Stop c4 can be served by two differents trip 3 and 5, so \(I(T_c) = I(T_4) = {3,5}
\subset {1,...25}\)
\end{enumerate}

\subsubsection{find best set of routes}
\label{sec:org136df00}
\begin{enumerate}
\item Variable
\label{sec:org024dd7f}
\(r_i\): binary whether trip i is selected \(\forall i \in \{1,...,m_S\}\)

\item Objective
\label{sec:org57ac947}
$$ min \hspace{0.5cm} \lambda \sum_{i=1}^{m_S} r_i + \sum_{i=1}^{m_S} r_i \Theta_i $$

\item Constraints
\label{sec:orgc9ed058}
 every stop must be served by at least one trip. 
$$\sum_{i \in I(T_c)} r_i \ge 1, \forall c \in C_S$$ 

where, for each trip \(i\) in \(T_s\), \(\Theta_i = \sum_{p:c_p \in C_i}
     \theta_p^{(i)}\), where \(C_i\) reprents the list of stops served by the
\(i-th\) trip \(i\),

\(\theta_p^{(i)}\) corresponds to tiem spent by student \(p\) on the bus druing
trip \(i\).

\(\lambda\) controls the importance of the number of trips relative to the
total time students on the bus.

In the optimal solution above, some stops may be served by more than one trip. 
Set of trips \(T_c^{opt} \subset T_S\) serving each such stop c. For each
\(i\in I(T_c^{opt})\), we compute the improvement \(\delta_i\) to the objective
that would be obtained by removing stop \(c\) from trip \(i\).
\end{enumerate}






\subsection{Scenario Selection}
\label{sec:org4f584c9}

\subsubsection{Node}
\label{sec:org09d2624}

\begin{center}
\begin{tabular}{lll}
Type & Notation & comment\\
\hline
depot & y & \\
trip & \(\rho_{S,h,R}\) & for each school S, and routing scenario \(1 \le h \le h_S\) and each trip \(R \in \mathcal{T}_S^h\)\\
available node & \(a_{S,h}\) & for each school S, and routing scenario \(1 \le h \le h_S\)\\
\end{tabular}
\end{center}

The set of arcs includes an arc from depot to trip; trip to corresponding
available node; from available node back to the depot.

Also, an arc from each available node to each trip node where trip \(R' \in
\mathcal{T}_{S'}^{h'}\) is compatible with the a bus staring from school S.
\(\tau_S + t_S^{drop-off} + t_{S, c_{R'}^{start}, \tau_S}^{drive} + \mathcal{T}_R \le \tau_{S'}\)

For a note \(i \in \mathcal{N}, let \hspace{0.1cm} \mathcal{I}{(i)} \subseteq \mathcal{N}\) be the
in-neighborhood of the node i, and \(\mathcal{O}{(i)} \subseteq \mathcal{N}\)

\subsubsection{Variable}
\label{sec:orgb930100}
integer flow variables \(f_i^j\) for each arc(i,j).

\subsubsection{Objective}
\label{sec:orge3b2392}
Minimize the total number of buses corresponds to minimizing the total flow
out of the yard node y.
$$ minimize \sum_{s \in S}\sum_{h=1}^{h_S}\sum_{R \in \mathcal{T}_S^h}f_y^{\rho_{S,h,R}}$$ 

\subsubsection{Constraints}
\label{sec:org377563a}
\begin{enumerate}
\item flows along all arcs must be integral.
\item $$\mathcal{Z}_{S,h} \in \{0,1\}, \forall S \in S, 1 \le h \le h_S$$
\item $$\sum_{h=1}^{h_S}Z_{S,h} = 1$$
\end{enumerate}

The binary variable \(Z_{S,h}\) is 1 if \(\mathcal{T}_S^h\) is the selected set
of trips for school S. \(3\). exactly one set of trips / one scenario is
selected for each school.
